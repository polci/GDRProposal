%\documentclass[paper=A4,11pt,DIV=12,headings=big]{scrartcl}
%\pdfoutput=1
%
%%%%% usepackage's %%%%
%\usepackage[a4paper, top=1.2in, bottom=1.2in, left=1.1in, right=1.1in]{geometry}
%\usepackage{amsfonts}
%\usepackage{amsmath,amssymb}
%\usepackage{mathrsfs}
%\usepackage[normal]{caption}
%\renewcommand{\captionfont}{\small}
%\renewcommand{\captionlabelfont}{\bfseries}
%
%\usepackage{cancel}
%\usepackage{xcolor}
%\usepackage{cite,bbm}
%\usepackage{enumerate}
%\usepackage{graphicx}
%\graphicspath{{./Figs/}}
%%\usepackage{multirow}
%\usepackage{wrapfig}
%
%%% abstract macros
%\renewenvironment{abstract}{%
%\begin{minipage}{0.95\textwidth}
%}
%{\par\noindent\end{minipage}}
%
%%% footnote macros
%\usepackage[multiple]{footmisc}
%\setlength{\footnotemargin}{2mm}
%\setlength{\footnotesep}{0.2cm}
%\let\oldfootnote\footnote\renewcommand\footnote[1]{\oldfootnote{\hspace{2mm}#1}}
%
%%% hyperrefs
%\pdfminorversion=5
%\definecolor{darkblue}{rgb}{0,0,0.9}
%\usepackage{hyperref}
%\hypersetup{
%linktocpage,
%colorlinks,
%citecolor=darkblue,
%filecolor=darkblue,
%linkcolor=darkblue,
%urlcolor=darkblue
%}
%%%%% end usepackage's %%%%
%
%%%%% macros %%%%
%\allowdisplaybreaks
%\addtokomafont{disposition}{\rmfamily\boldmath}
%\newcommand{\mailref}[1]{\href{mailto:#1}{#1}}
%
%\newcommand{\Ell}{\mathscr{L}}
%\newcommand{\HNP}{\mathscr{H}_{\rm NP}}
%\newcommand{\Op}{\mathcal{O}}
%\newcommand{\mc}{\mathcal}
%\renewcommand{\th}{{\tilde h}}
%\newcommand{\tH}{{\tilde H}}
%\newcommand{\tphi}{{\tilde \phi}}
%\newcommand{\<}{\langle}
%\renewcommand{\>}{\rangle}
%\newcommand{\Bsmumu}{\overline B_s \rightarrow \mu^+ \mu^-}
%\newcommand{\re}{{\rm Re}}
%\newcommand{\im}{{\rm Im}}
%\newcommand{\ra}{\rightarrow}
%\newcommand{\alem}{\alpha_{\rm em}}
%\newcommand{\sh}{\hat s}
%\renewcommand{\th}{\hat t}
%\newcommand{\uh}{\hat u}
%\newcommand{\mh}{\hat m}
%\newcommand{\Mh}{\hat M}
%
%\def\sla#1{\setbox0=\hbox{$#1$}\dimen0=\wd0
%      \setbox1=\hbox{/} \dimen1=\wd1 \ifdim\dimen0>\dimen1
%      \rlap{\hbox to \dimen0{\hfil/\hfil}} #1                        \else
%      \rlap{\hbox to \dimen1{\hfil$#1$\hfil}}
%      /   \fi}
%
%\newcommand{\be}{\begin{equation}}
%\newcommand{\ee}{\end{equation}}
%\newcommand{\bea}{\begin{eqnarray}}
%\newcommand{\eea}{\end{eqnarray}}
%\newcommand{\mb}{\mathbf}
%\newcommand{\nn}{\nonumber}
%
%\newcommand{\tmpbf}[1]{{\bf \boldmath [#1]}}
%\newcommand{\tmpbfdg}[1]{{\bf \boldmath \textcolor{red}{[DG: #1]}}}
%\newcommand{\tmpcite}{{\bf \textcolor{red}{[cite?]}}}
%\newcommand{\tmpred}[1]{\textcolor{red}{#1}}
%\newcommand{\tmpblue}[1]{\textcolor{blue}{#1}}
%%%%% macros %%%%
%
%\begin{document}
%
\section{Charm and Kaon Physics}

\subsection{Charm at the intensity frontier [V. Gligorov]}

Precise studies of the properties and decays of charmed hadrons are motivated as almost-null tests of the Standard Model (SM). In terms of CP violation, charm hadron decays involve only the first two generations of quarks, and CP violation in decay is therefore expected to occur at below the per mil level in the SM. Additionally, compared to beauty or strange hadrons, the mixing of neutral charm hadrons is slow, with both the $x = \Delta m/\Gamma$ and $y = \Delta \Gamma/ (2\Gamma)$ parameters at around the percent level. CP violation in charm decays has not been observed so far, and existing experimental limits are at the few per mil level. Theoretical predictions of charm CPV are difficult as long distance contributions dominate; CPV in decay close to the present experimental limits could be accommodate within the SM or could be signs of NP, and progress on the theory side will be required to disentangle the two. Similarly in the case of mixing, or CP violation in the interference of decay and mixing, more precise experimental results are needed to stimulate progress on the theoretical predictions.

In addition, charmed hadrons are also an interesting place to study rare and forbidden transitions, for example FCNS or lepton number violating decays. This is because charm hadrons are produced extremely copiously at the LHC (the $c \bar{c}$ cross-section is roughly 10\% of the total inelastic cross-section!), while their short but measurable decay times make them relatively simple to reconstruct and separate from background. Indeed within France, the only recent studies of charmed hadron decays were performed by the LAL-Orsay LHCb group, which studied the rare decays $D^0 \to \pi \mu\mu$ (with same sign muons) and $D^0 \to K \pi\mu\mu$. The former is of interest because the copious production rate of charmed hadrons allows effective limits to be placed on Majorana neutrinos. The latter is the charmed counterpart of $B\to K^*\mu\mu$ and has now been observed for the first time by LHCb, albeit within a dimuon $q^2$ region dominated by the omega and rho resonances. It should in principle share much of the same phenomenology of $B\to K^*\mu\mu$ once large signal yields become available, with the complication of much higher backgrounds from decays to hadronic resonances (such as KPirho) which subsequently decay to dimuon pairs.

In the upcoming period, the most critical work will be to improve the limits on CPV, both in decay and the interference of mixing and decay, as well as to make ever more precise measurements of charm mixing parameters using both the $D\to hh$ and $D\to K_s hh$ decay modes with the full Run II LHCb dataset. In addition, LHCb should obtain large samples of FCNC decays such as $D^0\to K \pi\mu\mu$, potentially allowing for an observation of the nonresonant (in the dimuon spectrum) decay and a measurement of angular observables similar to the ones which characterise $B\to K^*\mu\mu$. Finally, making more precise measurements of charm hadron lifetimes, in particular in the less well understood baryon sector, could aid the development of HQE tools and techniques required to eventually obtain precise SM predictions for mixing and CPV in the charm sector.

\subsection{New Kaon observables [D. Guadagnoli]}

Kaon mixing and decays belong traditionally to the most constraining processes for physics beyond the SM. In this section we provide motivations for searches of certain $K$ decays, in particular lepton-flavor violating (LFV) ones of the kind $K \to (\pi) e \mu$. These motivations rest to a good extent upon discrepancies found in recent LHCb and $B$-factory data. The most striking effect is in the quantity known as $R_K$ \cite{Aaij:2014ora} that, at face value, signals beyond-SM lepton flavor non-universality (LFNU). Interestingly, the effect is consistent in magnitude and size with the other discrepancies. Without further assumptions, LFNU at a non-SM level implies LFV at a non-SM level. In fact, to account for $R_K$ one needs to invoke new interactions distinguishing between leptons of different generations, for example lepton-lepton couplings with a new vector boson or quark-lepton couplings with a scalar leptoquark. The fermions involved in such interactions are generally not in the mass eigenbasis -- this basis doesn't even exist at the scale of these interactions, usually above the EWSB scale. After EWSB, rotation of the quark and lepton fields to the mass eigenbasis generates LFV effects along with the LFNU ones.

One theoretically appealing way to generate new-physics shifts of the required size is to invoke an effective interaction involving dominantly quarks and leptons of the third generation \cite{Glashow:2014iga}. Then, the amount of LNU pointed to by $R_K$ actually allows to quantify rather generally the expected amount of LFV to be in the ballpark of $10^{-8}$, which happens to be within reach at LHCb's run 2. This argument, reported in Refs. \cite{Guadagnoli:2016erb,Glashow:2014iga}, motivates searches of LFV $B$ decays as a promising direction at LHCb.

This very argument has implications in $K$ physics as well, in decays of the kind $K \to (\pi) \ell \ell'$, such as $K_L \to e^\pm \mu^\mp$ and $K^+ \to \pi^+ e^\pm \mu^\mp$. Experimental limits on these modes are more than ten years old: $\mc B(K_L \to e^\pm \mu^\mp) < 4.7 \times 10^{-12}$ \cite{Ambrose:1998us}, $\mc B(K^+ \to \pi^+ e^- \mu^+) < 1.3 \times 10^{-11}$ \cite{Sher:2005sp}, $\mc B(K^+ \to \pi^+ e^+ \mu^-) < 5.2 \times 10^{-10}$ \cite{Appel:2000tc}. Theoretical expectations for the above decays are straightforwardly calculable after suitably normalising the decay modes of interest in order to cancel phase-space factors. Defining $\beta^{(K)}$ as the ratio of the new-physics Wilson coefficient responsible for the decay in the numerator over the SM Wilson coefficient responsible for the normalising decay, we get
\bea
\label{eq:LFV_K_decays}
\frac{\Gamma(K_L \to e^\pm \mu^\mp)}{\Gamma(K^+ \to \mu^+ \nu_\mu)} = | \beta^{(K)} |^2~, \\
\label{eq:LFV_K_decays_2}
\frac{\Gamma(K^+ \to \pi^+ \mu^\pm e^\mp)}{\Gamma(K^+ \to \pi^0 \mu^+ \nu_\mu)} = 4 | \beta^{(K)} |^2~.
\eea

To get a numerical idea of the effects to be expected, we need a model predicting $|\beta^{(K)}|^2$. For the sake of definiteness, here we use ``model A'' of Ref. \cite{Guadagnoli:2015nra} (any other motivated model, for example Ref. \cite{Boucenna:2015raa}, will do), thereby obtaining $| \beta^{(K)} |^2 = 2.15 \times 10^{-14}$. Use of eqs. (\ref{eq:LFV_K_decays}) then implies
\be
\label{eq:KLemu}
\mc B(K_L \to e^\pm \mu^\mp) \approx 6 \times 10^{-14}~,
\ee
where we have used $\mc B(K^+ \to \mu^+ \nu_\mu) \approx 64\%$ and $\Gamma(K^+) / \Gamma(K_L) \approx 4.2$ \cite{Agashe:2014kda}. In addition
\be
\mc B(K^+ \to \pi^+ e^\pm \mu^\mp) \approx 3 \times 10^{-15}~.
\ee
after use of $\mc B(K^+ \to \pi^0 \mu^+ \nu_\mu) \approx 3\%$.

While the $K^+$ LFV mode is clearly too suppressed (within the considered model!), the $K_L$ one, eq. (\ref{eq:KLemu}), has a branching ratio close to $10^{-13}$. Such a rate may actually well be reachable at the NA62 experiment. As concerns LHCb, it should be noted that, although $K$ mesons are produced copiously, their lifetimes are typically too long for the detector size -- with the exception of the $K_S$. A dedicated study is thus necessary to understand the actual LHCb capabilities for the above decays.


\bibliographystyle{JHEP}
\bibliography{charm_and_K}

%\end{document}
