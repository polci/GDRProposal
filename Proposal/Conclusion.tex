\section{Conclusion}
The intensity frontier is a strategic approach to search for new physics.  Its validity is recognized at an international level and  it is a domain in which the French particle physics community is traditionally very active. Interesting and puzzling results are being produced by the experiments currently taking data, and further more are expected to come from the new generation of experiments that are starting or being planned. Theoretical progress are ensuring the precision of the currently performed tests, and additional clean observables are under investigation. 

The French community working in this field recognizes the need of coming together to pursue these searches together with a renewed enthusiasm  and with a stronger collaboration between the experimental and the theoretical laboratories in France.   
The GDR intensity frontier will be the place where we will be able to put together our experience, share our knowledge, renforce bounds and inspire new collaborations, ensuring that the French community continues to be competitive and focused on the most appealing topics of the field. It will additionally provide a forum to discuss the future of the field, and naturally promote the emergence of a young and dynamic generation of physicists. All this will allow to keep the current involvement and acquire an even higher visibility at a national and international level.

