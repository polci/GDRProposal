
\section{Introduction}

The "standard model" (SM) theory of particle physics has been very successful in describing the particles and their strong, electromagnetic and weak interactions. Nevertheless,  we know that it is not able to explain some aspects of the nature for which we have already experimental evidences. For example, it can not tell what dark matter and dark energy are, nor explain the hierarchy of the fermion masses, neither the matter-antimatter asymmetry in the universe. There is a  general consensus in the physics community that a theory more fundamental than the standard model should exist, generally referred to as "new physics" (NP). 

There are mainly two strategies  to search for new physics: direct and indirect searches.  In the direct search approach, experiments are built to try to create and detect new particles, for example  created during collisions at high energy, where we expect that new phenomena would occur. This is the main approach currently followed by the general purpose detectors, ATLAS and CMS, at the LHC. Instead, in the indirect approach,  one  looks  at processes for which theoretical predictions with uncertainties well under control  exist, and can be compared with  the experimental measurements:  observing a significant discrepancy between the two would be the sign of new physics. This technique is often applied to study processes mediated by loop and box diagrams.  Since particles in loops are virtual, yet undiscovered particles, even  with masses larger than the energy of the collisions, could intervene in the diagrams, modifying the rates and the properties of the decay respect to  the standard model predictions. These measurements need to be extremely precise, so they require to collect a large quantity of data.  A part from being a way to discover new physics, this approach is also  complementary  to the direct searches, providing constraints and highlights on the nature of the new physics and eventually on its flavour structure. 

Particle physics at the intensity frontier includes all those searches of new physics phenomena that, in order to probe weak couplings at higher energy scale, explore small cross section processes in which new physics is expected to show up. It largely makes use of the indirect approach, like in the case of flavor physics, but also uses the direct approach in some dedicated experiments,  like those searching for axions.  Experimentally, the challenge is to collect a very large and pure data sample to  put in evidence the rare processes. Theoretically, it is crucial to have under control the description of the processes in the SM framework. Theory and experiment then need to join together for a correct interpretation of the comparison of the experimental results with the theoretical predictions, to combine all the bounds produced in the different searches and to design a path towards the discovery of the new physics. 

Historically the French community has been very active in the intensity frontier field. From the experimental point of view, the main focus today is on the  LHCb experiment, dedicated to flavor physics and currently challenging the standard model predictions with many precise measurements. Worldwide, other experiments are currently using the same approach (NA62, MEG), some will start their data taking soon (Belle2) and other are in preparation. From the theoretical side, the community is also very active in this field, both on the interpretation of current data and in the reduction of theoretical systematic uncertainties in vue of the new experimental challenges to come.  Given the need of comparing directly  the theoretical predictions with the experimental measurements, the interplay between theory and experiment in this field is primordial. As a natural need of sharing competences and knowledge, during past years some collaborations have already risen between members of the two communities. A well known example of a fruitful exchange is the CKMfitter collaboration, which took origin from a French initiative. More recently, in the context of the study of rare $B$ meson decays, three  CNRS PEPS-PTI (Projet Exploratoire Premier Soutien de Physique Theorique et ses Interfaces) of one year each were proposed and accepted:  NouvPhyLHCb in 2014 and PhenoBas in 2015 and 2016. They allowed to organize three fruitful workshops,  allowing to establish first connections and collaborations between LHCb experimentalists and theorists  working on $b \to sll$ transitions. 

Following discussions among people active in the field, the need for a GDR in physics at the intensity frontier has been established. In fact, we believe that a more stable framework like the GDR would be beneficial to those working on high energy physics who are focussed on the intensity frontier, in order to bring this community together and renforce the interplay between the different research lines in the field. The role of the GDR would be to 
facilitate the collaboration between different laboratories and between theorists and experimentalists, with the purpose of keeping the community in touch and informed about the latest advancements in the field, exchanging ideas and spreading knowledge. In this way the GDR will stimulate the emergence of common projects within the French community and  allow it to grow. It will be a way to provide greater visibility of this large community on a national and international level. In addition, we believe that there is a real need to come together in order to discuss how research plans for the future should be shaped; including the decision of which experiments to become involved in.


We envisage the GDR to be divided into several working groups, which would both function independently and together. We have identified the following topics where there is currently activity and interest in the French community:
\begin{itemize}
\item {\bf CP violation in $B$ mesons.} Given its connections with the matter/antimatter asymmetry, CP violation is an interesting phenomenon by itself. Since the $B$-factories, the CP violation in the  $B$ sector has also been proven to be a precise test of the Standard Model, through the Unitarity Triangle measurement. This measurement is not yet completed, and LHCb and Belle2 will provide further insight on it, as well as additional tests involving the $B_s$ sector. 
\item {\bf Rare, radiative and semi-leptonic decays.} Mostly proceeding through loops, these decays are very powerful probe of new physics, provided that precise theoretical predictions meet clear experimental signatures. The large dataset collected by the LHCb experiment is currently showing the most exciting signs of slight deviations from the theoretical predictions that certainly deserves to be further analyses and deeply understood. 
\item {\bf Heavy flavour production and spectroscopy.} This field is the ideal framework to test the QCD predictions, which are crucial in many measurements looking for new physics. In addition it currently provides a novel evidence on the way in which quarks organise to form more complex structures, like tetraquarks and pentaquarks, whose existence is now established but not yet fully understood. 
\item {\bf Charm and Kaon physics.} The study of kaons and charmed mesons have been at the origin of the flavor physics. Given the present experimental opportunities, a renewed interest on the analysis of their decays is emerging, as they provide complementary ways to search for new physics effects. Although for the charm physics there is already a large production of data, for the kaons some experimental challenges need to be faced and additional theoretical observables are being proposed.  
\item {\bf Lepton Flavour. } (Neutrinos/Taonic/muonic decays/(g-2)). In the lepton flavor field some of the most interesting experimental data are suggesting that there is a need of revisiting the Standard Model. The experimental challenges here are big but the next generation of experiments should be able to allow studies on this. 
\item {\bf Future experiments.} (BelleII, FCC, SHIP). It will be very beneficial for our community to discuss about the future of our field, in a time where future upgrades of the LHCb experiment as well as new experiments are being proposed, in order to identify in which one we should focus our attention and participate in order to continue keeping an active role in the future. 
\end{itemize}
In the subsequent sections of this document we will provide a brief description to highlight  the interest of the working group topics,  summarising their  current status and the proposed near-future work of the French community.

The GDR would function through carefully planned workshops. More specifically, we plan to organize a general kick-off meeting, to bring the whole community together in order to define and consolidate collaborations and goals.  
This would be followed by a series of working group meetings, more intimate and focused on specific themes. These smaller meetings would allow detailed discussions and brainstorming within the specific topic of the working group, allowing close collaborations to emerge by really working together. Regularly, global workshops involving all the members of the GDR will be organized, where more general talks and discussions will be held and where we will ensure to share the advancements of the working groups and to address the connexions between them. This is particularly important since there is a clear interplay  between the different working groups. One of the purpose of the GDR will be also to have a wider look into what is done in the same field in other countries or in experiments where the French community is not currently directly involved, but which still represent an interest for the field. Presenting ourselves as a unified community, we will aim in establishing productive interactions inviting occasionally speakers from other experiments and other particle physics GDRs, ensuring in this way that we keep the connection with the whole field.

The format of these meetings would be free, and decided according to the specific needs and objectives, and it would be encouraged for younger members of the community to participate in the organisation and the discussion.
In fact, we further hope to use the GDR as an opportunity to put the younger members of the community in the spotlight. One way to do so will be  by giving to the postdocs the responsibility of organising and chairing the meetings. In addition, we will create an environment where PhD students would feel confident to present their work and interact with physicists from other laboratories.










